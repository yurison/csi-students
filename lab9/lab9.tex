\documentclass[a4paper, 12pt]{article}
\usepackage[utf8]{inputenc}
\usepackage[english,russian]{babel}
\usepackage[T1, T2A]{fontenc}
\usepackage{graphicx}

\usepackage{pgfplots}
\usetikzlibrary{pgfplots.polar}
\pgfplotsset{compat=1.13, grid=major}
\usepackage[left = 2cm, right = 2cm, bottom = 2cm, top = 2cm]{geometry}
\usepackage[top=2cm, left=2cm, right=2cm, left=2cm]{geometry}
\usepackage{amsmath}

\usepackage{tabu}
\usepackage{threeparttablex} 
\usepackage{booktabs} 
\usepackage[tableposition=top]{caption}

\usepackage{subcaption}
\DeclareCaptionLabelFormat{gostfigure}{Рисунок #2}
\DeclareCaptionLabelFormat{gosttable}{Таблица #2}
\DeclareCaptionLabelSeparator{gost}{~---~}
\captionsetup{labelsep=gost}
\captionsetup[figure]{labelformat=gostfigure}
\captionsetup[table]{labelformat=gosttable}
\renewcommand{\thesubfigure}{\asbuk{subfigure}}
\captionsetup[table]{labelformat=simple, labelsep = endash, justification = raggedright, singlelinecheck = off}
\usepackage{indentfirst}
\graphicspath{{image/}}
\newcommand\tline[2]{$\underset{\text{#1}}{\text{\underline{\hspace{#2}}}}$}

% PGFPlots Table ========================================================
\usepackage{pgfplotstable}
\renewcommand{\arraystretch}{1.5}
% recommended:
\usepackage{booktabs}
\usepackage{colortbl}
% pgfplotstable settings
\pgfplotstableset{
    columns/w/.style = {column name = {\boldmath$\omega$}, column type = |c},
    columns/logW/.style = {column name = {\boldmath$\lg{\omega}$}, column type = |c},
    columns/A/.style = {column name = {\boldmath$A(\omega)$}, column type = |c},
    columns/logA/.style = {column name = {\boldmath$20\lg{A(\omega)}$}, column type = |c},
    columns/psi/.style = {column name = {\boldmath$\psi$}, column type = |c|},
    every head row/.style = {before row = \hline},
    after row = {[1mm] \hline},
}

\usepackage{amsmath}
\usepackage{float}

% Table and figure setting ==============================================
\usepackage{threeparttable}
%Change label separator
\usepackage{caption}
\captionsetup[table]{labelformat=simple, labelsep = endash, justification = raggedright, singlelinecheck = off}
\captionsetup[figure]{labelformat=simple, labelsep = endash, name = Рисунок}

% Paragraph indent
\usepackage{indentfirst}
\setlength{\parindent}{15mm}




\begin{document}
	\begin{titlepage}
	\centering
	{\fontsize{12pt}{5cm}\selectfont \bfseries Министерство образования и науки Российской Федерации} \\ \vspace{0.5cm}
	{\fontsize{7pt}{5cm}\selectfont ФЕДЕРАЛЬНОЕ ГОСУДАРСТВЕННОЕ АВТОНОМНОЕ ОБРАЗОВАТЕЛЬНОЕ УЧРЕЖДЕНИЕ ВЫСШЕГО ПРОФЕССИОНАЛЬНОГО ОБРАЗОВАНИЯ} \\ 
	\vspace{1cm}
	{\fontsize{12pt}{5cm}\selectfont \bfseries САНКТ-ПЕТЕРБУРГСКИЙ УНИВЕРСИТЕТ ИНФОРМАЦИОННЫХ ТЕХНОЛОГИЙ, МЕХАНИКИ И ОПТИКИ} \\ \vspace{1.5cm}
	
	{\fontsize{14pt}{5cm}\selectfont Кафедра \hspace{1cm} \underline{Систем Управления и Информатики}  \hspace{1cm} Группа \underline{Р3340}} \\ 
	\vspace{2cm}
	
	{\fontsize{20pt}{5cm}\selectfont \bfseries Лабораторная работа №9} \\
	{\fontsize{20pt}{5cm}\selectfont \bfseries “Экспериментальное построение частотных характеристик типовых динамических звеньев”} \\
	{\fontsize{14pt}{5cm}\selectfont Вариант - 11} \\
	\vspace{1.5cm}
	
	\flushleft
	
	{Выполнил \hspace{2cm} \tline{(фамилия, и.о.)}{9cm} (подпись)} \\
	\vspace{2cm}
	
	{Проверил \hspace{2cm} \tline{(фамилия, и.о.)}{9cm} (подпись)} \\
	\vspace{5cm}
	
	"\underline{\hspace{0.7cm}}"\hspace{0.2cm}\underline{\hspace{2cm}}\hspace{0.2cm}20\underline{\hspace{0.7cm}}г. \hspace{2cm} Санкт-Петербург, \hspace{2cm} 20\underline{\hspace{0.7cm}}г. \\ \vspace{1cm}
	
	Работа выполнена с оценкой \hspace{1cm} \underline{\hspace{8cm}} \\ 
	\vspace{1cm}
	Дата защиты "\underline{\hspace{0.7cm}}"\hspace{0.2cm}\underline{\hspace{2cm}}\hspace{0.2cm}20\underline{\hspace{0.7cm}}г.
	
\end{titlepage}

\begin{center}
	\section*{Задание}
\end{center}
\textbf{Цель работы} - изучение частотных характеристик типовых динамических звеньев и способов их построения. \par
Если на вход устойчивого линейного звена с передаточной функцией $W(s)$ подается гармонический сигнал $g(t) = g_m\sin{\omega t}$, то на его выходе в установившемся режиме будет гармонический сигнал $y(t) = y_m\sin{(\omega t + \psi)}$. Задача состоим в том, чтобы определить зависимость амплитуды $y_m(\omega)$ и разности фаз между выходым сигналом и входным $\psi(\omega)$ от частоты входного сигнала. Полученные графики получили название: амплитудно-частотная характеристика (АЧХ) и фазо-частотная характеристика (ФЧХ). \par
В данной реботе необходимо получить АЧХ и ФЧХ линейных динамических звеньев, представленных в таблице 1, подставив в них параметры, указанные в таблице 2. После чего, на основе двух предыдущих характеристик, построить амплитудно-фазовую характеристику (АФЧХ).

\begin{table}[h!]
	%\tabulinesep = 2mm
	\centering
	\begin{threeparttable}
		\caption{Исходные элементарные звенья}
		\begin{tabular} {|l|c|}
			\hline
			Тип звена & Передаточная функция \\ [0.5cm]  \hline
			Идеальное интегрирующее    & $\displaystyle\frac{k}{s}$ \\ [0.5cm]  \hline
			Изодромное & $\displaystyle\frac{k(Ts + 1)}{s}$ \\ [0.5cm]  \hline
			Дифференцирующее с замедлением & $\displaystyle\frac{ks}{Ts + 1}$ \\ [0.5cm] \hline
		\end{tabular}
	\end{threeparttable} 
\end{table}

\begin{table}[h!]
	\tabulinesep = 2mm
	\centering
	\begin{threeparttable}
		\caption{Параметры}\label{tab:perflogcross}
		\begin{tabular}{|c|c|c|}
			\hline
			K & T & $\xi$ \\ \hline
			5 & 0.1 & 0.1 \\
			\hline
		\end{tabular}
	\end{threeparttable} 
\end{table}


\newpage
\begin{center}
	\section{Исследование идеального интегрирующего звена}
\end{center}

В таблице 3 представлены данные, снятые по графикам переходных процессов. \par
\begin{table}[h!]
    \centering
    \begin{threeparttable}
        \caption{Полученные данные} \label{tab:perflogcross}
        \pgfplotstabletypeset[]{data/data4.txt}
    \end{threeparttable}
\end{table}
Передаточная функия исследуемого звена представлена в таблице 1. Из нее можно построить частотную функцию и найти выражения для АЧХ и ФЧХ.
   $$W(j\omega) = W(s)\big|_{s = j\omega} = \frac{k}{j\omega} = \frac{jk\omega}{-\omega^2} = -j\frac{k}{\omega} \eqno (1)$$
$$U(\omega) = 0 \eqno (2)$$
$$V(\omega) = \frac{k}{\omega} = \frac{5}{\omega} \eqno (3)$$
$$A(\omega) = \frac{5}{\omega} \eqno (4)$$
$$L(\omega) = 20\lg{A(\omega)} = 20\lg{\frac{5}{\omega}} = 20\lg{5} - 20\lg{\omega} \eqno(5)$$
$$\psi(\omega) = \arctg{\frac{V(\omega)}{U(\omega)}} = -\arctg{\frac{5}{\omega}} = -\arctg{\infty} = -\frac{\pi}{2} \eqno(6)$$

\newpage
Экспериментально построенные характеристики данного звена представлены ниже.
\begin{figure}[h!]
    \begin{subfigure}{0.5\textwidth}
        \centering
        \begin{tikzpicture}
            \begin{semilogxaxis} [
                    width = 0.9\textwidth,
                    xlabel = {$\omega$, 1/c},
                    ylabel = {$A(\omega)$},
                ]
                \addplot table [x={w}, y={A}] {data/data4.txt};
            \end{semilogxaxis}
        \end{tikzpicture}
        \caption{График АЧХ}
    \end{subfigure}
    \begin{subfigure}{0.5\textwidth}
        \centering
        \begin{tikzpicture}
            \begin{semilogxaxis} [
                    width = 0.9\textwidth,
                    xlabel = {$\omega$, 1/c},
                    ylabel = {$\psi$, градусы},
                ]
                \addplot table [x={w}, y={psi}] {data/data4.txt};
            \end{semilogxaxis}
        \end{tikzpicture}
        \caption{График ФЧХ}
    \end{subfigure}
    
    \vspace{0.5cm}

    \begin{subfigure}{0.5\textwidth}
        \centering
        \begin{tikzpicture}
            \begin{polaraxis} [
                    width = 0.9\textwidth,
                    xlabel = {$A(\omega)$},
                    ylabel = {$\psi$, градусы},
                ]
                \addplot table [x={psi}, y={A}] {data/data4.txt};
            \end{polaraxis}
        \end{tikzpicture}
        \caption{График АФЧХ}
    \end{subfigure}
    \begin{subfigure}{0.5\textwidth}
        \centering
        \begin{tikzpicture}
            \begin{polaraxis} [
                    width = 0.9\textwidth,
                    xlabel = {$L_m$, дБ},
                    ylabel = {$\psi$, градусы},
                ]
                \addplot table [x={psi}, y={logA}] {data/data4.txt};
            \end{polaraxis}
        \end{tikzpicture}
        \caption{График ЛАФЧХ}
    \end{subfigure}
    \caption{Частотные характеристики идеального интегрирующего звена}
\end{figure}

\newpage
\begin{center}
	\section{Исследование изодромного звена}
\end{center}
В таблице 4 представлены данные, снятые по графикам переходных процессов.
\begin{table}[h!]
    \centering
    \begin{threeparttable}
        \caption{Полученные данные}
        \pgfplotstabletypeset[]{data/data5.txt}
    \end{threeparttable}
\end{table}

Передаточная функия исследуемого звена представлена в таблице 1. Из нее можно построить частотную функцию и найти выражения для АЧХ и ФЧХ.
$$W(j\omega) = W(s)\big|_{s = j\omega} = \frac{k(1 + jT\omega)}{j\omega} = \frac{-kT\omega^2 + jk\omega}{-\omega^2} = \frac{kT\omega - jk}{\omega} = kT - j\frac{k}{\omega} \eqno (7)$$
$$U(\omega) = kT = 0.5 \eqno (8)$$
$$V(\omega) = -\frac{k}{\omega} = -\frac{5}{\omega} \eqno (9)$$
$$A(\omega) = \sqrt{0.25 + \displaystyle{\frac{25}{\omega^2}}} \eqno (10)$$
$$L(\omega) = 20\lg{\sqrt{1 + \frac{4}{\omega^2}}} \eqno(11)$$
$$\psi(\omega) = -\arctg{\frac{10}{\omega}} \eqno (12)$$
\newpage
\begin{figure}[h!]
    \begin{subfigure}{0.5\textwidth}
        \centering
        \begin{tikzpicture}
            \begin{semilogxaxis} [
                    width = 0.9\textwidth,
                    xlabel = {$\omega$, 1/c},
                    ylabel = {$A(\omega)$},
                ]
                \addplot table [x={w}, y={A}] {data/data5.txt};
            \end{semilogxaxis}
        \end{tikzpicture}
        \caption{График АЧХ}
    \end{subfigure}
    \begin{subfigure}{0.5\textwidth}
        \centering
        \begin{tikzpicture}
            \begin{semilogxaxis} [
                    width = 0.9\textwidth,
                    xlabel = {$\omega$, 1/c},
                    ylabel = {$\psi$, градусы},
                ]
                \addplot table [x={w}, y={psi}] {data/data5.txt};
            \end{semilogxaxis}
        \end{tikzpicture}
        \caption{График ФЧХ}
    \end{subfigure}
    
    \vspace{0.5cm}

    \begin{subfigure}{0.5\textwidth}
        \centering
        \begin{tikzpicture}
            \begin{polaraxis} [
                    width = 0.9\textwidth,
                    xlabel = {$A(\omega)$},
                    ylabel = {$\psi$, градусы},
                ]
                \addplot table [x={psi}, y={A}] {data/data5.txt};
            \end{polaraxis}
        \end{tikzpicture}
        \caption{График АФЧХ}
    \end{subfigure}
    \begin{subfigure}{0.5\textwidth}
        \centering
        \begin{tikzpicture}
            \begin{polaraxis} [
                    width = 0.9\textwidth,
                    xlabel = {$L_m$, дБ},
                    ylabel = {$\psi$, градусы},
                ]
                \addplot table [x={psi}, y={logA}] {data/data5.txt};
            \end{polaraxis}
        \end{tikzpicture}
        \caption{График ЛАФЧХ}
    \end{subfigure}
    \caption{Частотные характеристики изодромного звена}
\end{figure}

\newpage
\begin{center}
	\section{Исследование дифференцирующего звена с замедлением}
\end{center}
В таблице 4 представлены данные, снятые по графикам переходных процессов. \par
\begin{table}[h!]
    \centering
    \begin{threeparttable}
        \caption{Полученные данные}
        \pgfplotstabletypeset[]{data/data7.txt}
    \end{threeparttable}
\end{table}

Ниже представлены выражения частотных характеристик.


$$W(jw) = \frac{{5jw}}{{0.1jw + 1}} = \frac{{5jw(1 - 0.1jw)}}{{1 + 0.01{w^2}}} = \frac{{0.5{w^2}}}{{1 + 0.01{w^2}}} + \frac{{5w}}{{1 + 0.01{w^2}}}j\eqno (13)$$ \\
$$U(w) = \frac{{0.5{w^2}}}{{1 + 0.01{w^2}}}\eqno (14)$$ \\
$$V(w) = \frac{{5w}}{{1 + 0.01{w^2}}}\eqno (15)$$ \\
$$A(w) = \sqrt {{U^2} + {V^2}}  = \sqrt {\frac{{0.25{w^4}}}{{{{(1 + 0.01{w^2})}^2}}} + \frac{{25{w^2}}}{{{{(1 + 0.01{w^2})}^2}}}}  = \frac{{5w}}{{\sqrt {1 + 0.01{w^2}} }}\eqno (16)$$ \\
$$\psi(w) = arctg\frac{{V(w)}}{{U(w)}} = arctg\frac{{10}}{w}\eqno (17)$$




На рисунке 3 представлены графики по данным, которые были сняты графически и полученные аналитически из выражения (8). Как видно из графиков, чем меньше частота колебаний - тем меньше амплитуда собственных колебаний системы.
\newpage
\begin{figure}[h!]
    \begin{subfigure}{0.5\textwidth}
        \centering
        \begin{tikzpicture}
            \begin{semilogxaxis} [
                    width = 0.9\textwidth,
                    xlabel = {$\omega$, 1/c},
                    ylabel = {$A(\omega)$},
                ]
                \addplot table [x={w}, y={A}] {data/data7.txt};
      
            \end{semilogxaxis}
        \end{tikzpicture}
        \caption{График АЧХ}
    \end{subfigure}
    \begin{subfigure}{0.5\textwidth}
        \centering
        \begin{tikzpicture}
            \begin{semilogxaxis} [
                    width = 0.9\textwidth,
                    xlabel = {$\omega$, 1/c},
                    ylabel = {$\psi$, градусы},
                ]
                \addplot table [x={w}, y={psi}] {data/data7.txt};
                
            \end{semilogxaxis}
        \end{tikzpicture}
        \caption{График ФЧХ}
    \end{subfigure}
    
    \vspace{0.5cm}

    \begin{subfigure}{0.5\textwidth}
        \centering
        \begin{tikzpicture}
            \begin{polaraxis} [
                    width = 0.9\textwidth,
                    xlabel = {$A(\omega)$},
                    ylabel = {$\psi$, градусы},
                ]
                \addplot table [x={psi}, y={A}] {data/data7.txt};
                
            \end{polaraxis}
        \end{tikzpicture}
        \caption{График АФЧХ}
    \end{subfigure}
    \begin{subfigure}{0.5\textwidth}
        \centering
        \begin{tikzpicture}
            \begin{polaraxis} [
                    width = 0.9\textwidth,
                    xlabel = {$L_m$, дБ},
                    ylabel = {$\psi$, градусы},
                ]
                \addplot table [x={psi}, y={logA}] {data/data7.txt};
                
            \end{polaraxis}
        \end{tikzpicture}
        \caption{График ЛАФЧХ}
    \end{subfigure}
    \caption{Частотные характеристики дифференцирующего звена с замедлением }
\end{figure}

\newpage
\begin{center}
	\section{Асимптотические ЛАЧХ исследуемых звеньев}
\end{center}
\begin{figure}[h]
	
		\centering
		\begin{tikzpicture}
		\begin{semilogxaxis} [
		width = 0.9\textwidth,
		xlabel = {$\omega$, 1/c},
		ylabel = {$L(\omega)$, Дб},
		]
		\addplot table [x={w}, y={logA}] {data/data7.txt};
		\draw (1,14) --(10, 34) --(100,34);
		\end{semilogxaxis}
		\end{tikzpicture}
		\caption{График ЛАЧХ дифференцирующего звена с замедлением }
\end{figure}
	
\begin{figure}
	\centering
	\begin{tikzpicture}
	\begin{semilogxaxis} [
	width = 0.9\textwidth,
	xlabel = {$\omega$, 1/c},
	ylabel = {$L(\omega)$, Дб},
	]
	\addplot table [x={w}, y={logA}] {data/data4.txt};
	\draw (1,14) --(100,-26);
	\end{semilogxaxis}
	\end{tikzpicture}
	\caption{График ЛАЧХ идеального интегрирующего }
\end{figure}

\begin{figure}
	\centering
	\begin{tikzpicture}
	\begin{semilogxaxis} [
	width = 0.9\textwidth,
	xlabel = {$\omega$, 1/c},
	ylabel = {$L(\omega)$, Дб},
	]
	\addplot table [x={w}, y={logA}] {data/data5.txt};
	\draw (1,14) --(10, -6) --(100,-6);
	\draw[dashed] (0.01, 23.52) -- (0.2, 23.52);
	\draw[dashed] (0.2, 23.52) -- (0.2, 0);
	\end{semilogxaxis}
	\end{tikzpicture}
	\caption{График ЛАЧХ изодромного звена }
\end{figure}


\newpage
\begin{center}
	\section*{Выводы}
\end{center}

В данной работе мы исследовали частотные характеристике трех звеньев: Идеального интегрирующего, изодромного, и дифференцирующего с замедлением. Получили экспериментально графики частотных характеристик и сравнили их с соответствующими выражениями. \par
Как видно из рисунка 4 и 6, при частоте $\omega_c = 1/T = 10$ ЛАЧХ изменяет наклон, что соответствует теории (наклон дифференцирующего звена с замедление изменяется с $+20$ Дб/дек до $0$ Дб/дек, изодромного соответсвенно от $-20$ Дб/дек до $0$ Дб/дек) и у идеального интегрирующего не изменяет


\end{document}